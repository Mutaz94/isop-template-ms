% Options for packages loaded elsewhere
\PassOptionsToPackage{unicode,linktoc=all,pdfpagemode=FullScreen,bookmarks=true}{hyperref}
\PassOptionsToPackage{hyphens}{url}
\PassOptionsToPackage{dvipsnames,svgnames,x11names}{xcolor}
%
\documentclass[
  letterpaper,
  toc=chapterentrywithdots,
  11pt,
  headings=small]{scrreprt}

\usepackage{amsmath,amssymb}
\usepackage{iftex}
\ifPDFTeX
  \usepackage[T1]{fontenc}
  \usepackage[utf8]{inputenc}
  \usepackage{textcomp} % provide euro and other symbols
\else % if luatex or xetex
  \usepackage{unicode-math}
  \defaultfontfeatures{Scale=MatchLowercase}
  \defaultfontfeatures[\rmfamily]{Ligatures=TeX,Scale=1}
\fi
\usepackage{lmodern}
\ifPDFTeX\else  
    % xetex/luatex font selection
  \setmainfont[Path=./engine/misc/Times/,Extension=.ttf,UprightFont=*,BoldFont=*
Bold]{Times New Roman}
  \setsansfont[Path=./engine/misc/Times/,Extension=.ttf,UprightFont=*,BoldFont=*
Bold]{Times New Roman}
\fi
% Use upquote if available, for straight quotes in verbatim environments
\IfFileExists{upquote.sty}{\usepackage{upquote}}{}
\IfFileExists{microtype.sty}{% use microtype if available
  \usepackage[]{microtype}
  \UseMicrotypeSet[protrusion]{basicmath} % disable protrusion for tt fonts
}{}
\makeatletter
\@ifundefined{KOMAClassName}{% if non-KOMA class
  \IfFileExists{parskip.sty}{%
    \usepackage{parskip}
  }{% else
    \setlength{\parindent}{0pt}
    \setlength{\parskip}{6pt plus 2pt minus 1pt}}
}{% if KOMA class
  \KOMAoptions{parskip=half}}
\makeatother
\usepackage{xcolor}
\usepackage[margin=.75in,left=1in,right=1in,top=1in,bottom=3cm,footskip=2cm,marginparwidth=1in,marginparsep=0.75cm]{geometry}
\setlength{\emergencystretch}{3em} % prevent overfull lines
\setcounter{secnumdepth}{5}
% Make \paragraph and \subparagraph free-standing
\ifx\paragraph\undefined\else
  \let\oldparagraph\paragraph
  \renewcommand{\paragraph}[1]{\oldparagraph{#1}\mbox{}}
\fi
\ifx\subparagraph\undefined\else
  \let\oldsubparagraph\subparagraph
  \renewcommand{\subparagraph}[1]{\oldsubparagraph{#1}\mbox{}}
\fi


\providecommand{\tightlist}{%
  \setlength{\itemsep}{0pt}\setlength{\parskip}{0pt}}\usepackage{longtable,booktabs,array}
\usepackage{calc} % for calculating minipage widths
% Correct order of tables after \paragraph or \subparagraph
\usepackage{etoolbox}
\makeatletter
\patchcmd\longtable{\par}{\if@noskipsec\mbox{}\fi\par}{}{}
\makeatother
% Allow footnotes in longtable head/foot
\IfFileExists{footnotehyper.sty}{\usepackage{footnotehyper}}{\usepackage{footnote}}
\makesavenoteenv{longtable}
\usepackage{graphicx}
\makeatletter
\def\maxwidth{\ifdim\Gin@nat@width>\linewidth\linewidth\else\Gin@nat@width\fi}
\def\maxheight{\ifdim\Gin@nat@height>\textheight\textheight\else\Gin@nat@height\fi}
\makeatother
% Scale images if necessary, so that they will not overflow the page
% margins by default, and it is still possible to overwrite the defaults
% using explicit options in \includegraphics[width, height, ...]{}
\setkeys{Gin}{width=\maxwidth,height=\maxheight,keepaspectratio}
% Set default figure placement to htbp
\makeatletter
\def\fps@figure{htbp}
\makeatother
\newlength{\cslhangindent}
\setlength{\cslhangindent}{1.5em}
\newlength{\csllabelwidth}
\setlength{\csllabelwidth}{3em}
\newlength{\cslentryspacingunit} % times entry-spacing
\setlength{\cslentryspacingunit}{\parskip}
\newenvironment{CSLReferences}[2] % #1 hanging-ident, #2 entry spacing
 {% don't indent paragraphs
  \setlength{\parindent}{0pt}
  % turn on hanging indent if param 1 is 1
  \ifodd #1
  \let\oldpar\par
  \def\par{\hangindent=\cslhangindent\oldpar}
  \fi
  % set entry spacing
  \setlength{\parskip}{#2\cslentryspacingunit}
 }%
 {}
\usepackage{calc}
\newcommand{\CSLBlock}[1]{#1\hfill\break}
\newcommand{\CSLLeftMargin}[1]{\parbox[t]{\csllabelwidth}{#1}}
\newcommand{\CSLRightInline}[1]{\parbox[t]{\linewidth - \csllabelwidth}{#1}\break}
\newcommand{\CSLIndent}[1]{\hspace{\cslhangindent}#1}

\usepackage{tabularray}
\usepackage{booktabs}
\usepackage{tabularx}
\usepackage{mdframed}
\usepackage[acronym,toc,nomain]{glossaries}
\usepackage{todonotes}
\usepackage{threeparttable}

\RedeclareSectionCommand[
  beforeskip=0pt,
  afterindent=false% <- added
]{chapter}
\setkomafont{chapter}{\fontsize{14}{16.8}\selectfont}
\setkomafont{section}{\fontsize{12}{14}\selectfont}   
\setkomafont{subsection}{\fontsize{12}{14}\selectfont}
%\usepackage{tocloft}

\newglossarystyle{mystyle}
{
    \setglossarystyle{long3colheader}%
    \renewcommand*{\glossaryheader}{
    %  \textbf{Notation} & \textbf{Description} &
      \vspace{8pt}
    \endhead}%
    \renewcommand*{\glossaryname}{}
    \renewcommand*{\entryname}{}
    \renewcommand*{\pagelistname}{}
   % \renewcommand*{\descriptionname}{\textbf{Description}}
    \renewcommand{\glossentry}[2]{%
       \glsentryitem{##1}\glstarget{##1}{\glossentryname{##1}}
        & \glossentrydesc{##1}
        & ##2
        \tabularnewline*}%
}
\makeatletter
\makeatother
\makeatletter
\@ifpackageloaded{bookmark}{}{\usepackage{bookmark}}
\makeatother
\makeatletter
\@ifpackageloaded{caption}{}{\usepackage{caption}}
\AtBeginDocument{%
\ifdefined\contentsname
  \renewcommand*\contentsname{Table of contents}
\else
  \newcommand\contentsname{Table of contents}
\fi
\ifdefined\listfigurename
  \renewcommand*\listfigurename{Figures in-text}
\else
  \newcommand\listfigurename{Figures in-text}
\fi
\ifdefined\listtablename
  \renewcommand*\listtablename{Tables in-text}
\else
  \newcommand\listtablename{Tables in-text}
\fi
\ifdefined\figurename
  \renewcommand*\figurename{Figure}
\else
  \newcommand\figurename{Figure}
\fi
\ifdefined\tablename
  \renewcommand*\tablename{Table}
\else
  \newcommand\tablename{Table}
\fi
}
\@ifpackageloaded{float}{}{\usepackage{float}}
\floatstyle{ruled}
\@ifundefined{c@chapter}{\newfloat{codelisting}{h}{lop}}{\newfloat{codelisting}{h}{lop}[chapter]}
\floatname{codelisting}{Listing}
\newcommand*\listoflistings{\listof{codelisting}{List of Listings}}
\makeatother
\makeatletter
\@ifpackageloaded{caption}{}{\usepackage{caption}}
\@ifpackageloaded{subcaption}{}{\usepackage{subcaption}}
\makeatother
\makeatletter
\@ifpackageloaded{tcolorbox}{}{\usepackage[skins,breakable]{tcolorbox}}
\makeatother
\makeatletter
\@ifundefined{shadecolor}{\definecolor{shadecolor}{rgb}{.97, .97, .97}}
\makeatother
\makeatletter
\makeatother
\makeatletter
\makeatother
\ifLuaTeX
  \usepackage{selnolig}  % disable illegal ligatures
\fi
\IfFileExists{bookmark.sty}{\usepackage{bookmark}}{\usepackage{hyperref}}
\IfFileExists{xurl.sty}{\usepackage{xurl}}{} % add URL line breaks if available
\urlstyle{same} % disable monospaced font for URLs
\hypersetup{
  pdftitle={Population Pharmacokinetic Report},
  pdfauthor={MMJ},
  colorlinks=true,
  linkcolor={blue},
  filecolor={Maroon},
  citecolor={Blue},
  urlcolor={Blue},
  pdfcreator={LaTeX via pandoc}}

\usepackage[headsepline,footsepline]{scrlayer-scrpage}
\usepackage{cleveref}
\pagestyle{scrheadings}
\lohead{Population PK Report \\ Report No.: }
\rohead{DRAFT}
\ifoot{CONFIDENTIAL}
\ofoot{19 Aug 2024} % empty 
\cfoot{Page \thepage}

\renewcommand*\chapterpagestyle{scrheadings}

\renewcommand{\glsnamefont}[1]{#1}

\setlength\LTleft{0pt}
\setlength\LTright{0pt}
\setlength\glsdescwidth{0.8\hsize}
\loadglsentries{}
\setglossarystyle{mystyle}
\makenoidxglossaries
\begin{document}
% TODO: Add custom LaTeX header directives here
\thispagestyle{empty}
\vskip2cm
{\centerline{\textbf{POPULATION PHARMACOKINETICS REPORT}}}
\vskip1cm
\begin{table}[!h]
      \setlength{\tabcolsep}{5pt}
      \renewcommand{\arraystretch}{1.5}
      \begin{tabularx}{\textwidth}{|p{0.3\textwidth}|X|}
            \hline
            \textbf{Report Number:} & \\
            \hline
            \textbf{Report Title:} & Population Pharmacokinetic
Report \\
            \hline
            \textbf{Study Drug:} &  Drug A\\
            \hline
            \textbf{Indication(s):} & Nothing\\
            \hline
            \textbf{Study Number(s):} & 12345\\
            \hline
            \textbf{Sponsor:} & \\
            \hline
            \textbf{Prepared By:} & MMJ\\
            \hline
            \textbf{Reviewed By:} & MMJ \\
            \hline
            \textbf{Approved By:} & MMJ\\
            \hline
            \textbf{Report Date:} & 19 Aug 2024\\
            \hline
            \textbf{Report Status:}  & DRAFT\\
            \hline
      \end{tabularx}
\end{table}
\vskip0.5cm
\begin{center}
{\textbf{CONFIDENTIAL}}
\begin{mdframed}
    {\normalsize This is a Company Name. document that contains
confidential information. It is intended solely for the recipient and
must not be disclosed to any other party. This material may be used only
for evaluating or conducting clinical investigations; any other proposed
use requires written consent from Company Name.}
\end{mdframed}
\end{center}
\ifdefined\Shaded\renewenvironment{Shaded}{\begin{tcolorbox}[sharp corners, interior hidden, borderline west={3pt}{0pt}{shadecolor}, boxrule=0pt, breakable, enhanced, frame hidden]}{\end{tcolorbox}}\fi

\renewcommand*\contentsname{Contents}
{
\hypersetup{linkcolor=blue}
\setcounter{tocdepth}{2}
\tableofcontents
}
\listoffigures
\listoftables
\bookmarksetup{startatroot}

\hypertarget{section}{%
\chapter{}\label{section}}

\bookmarksetup{startatroot}

\hypertarget{sec-executive-summary}{%
\chapter{EXECUTIVE SUMMARY}\label{sec-executive-summary}}

\textit{{The executive summary provides a brief overview of the analysis, highlighting the purpose, key findings affecting usage, labeling, and any additional recommendations based on the population PK analysis. It should be concise, typically consisting of 2-3 short paragraphs, and prepared with the understanding that it may be copied and pasted into a regulatory document.}}

\bookmarksetup{startatroot}

\hypertarget{sec-synobsis}{%
\chapter*{SYNOBSIS}\label{sec-synobsis}}
\addcontentsline{toc}{chapter}{SYNOBSIS}

\markboth{SYNOBSIS}{SYNOBSIS}

\bookmarksetup{startatroot}

\hypertarget{sec-introduction}{%
\chapter{INTRODUCTION}\label{sec-introduction}}

\bookmarksetup{startatroot}

\hypertarget{sec-methods}{%
\chapter{METHODS}\label{sec-methods}}

\bookmarksetup{startatroot}

\hypertarget{sec-results}{%
\chapter{RESULTS}\label{sec-results}}

For each analysis (e.g.~PopPK, PK/PD analysis, exposure-response
analysis and simulations), an own sub-section should be included.

\hypertarget{exploratory-data-analysis}{%
\section{Exploratory Data Analysis}\label{exploratory-data-analysis}}

Exploratory analysis performed prior to modeling,
e.g.~concentration-time curves or descriptive analyses of PD measures.

\hypertarget{model-development}{%
\section{Model Development}\label{model-development}}

\hypertarget{base-model}{%
\subsection{Base Model}\label{base-model}}

Structural models considered, and model chosen. Diagnostic Plots for the
assessment of structural models attempted. Discussion of alternative
residual error and between subject variability models attempted,
including alternative forms of variance covariance matrices as well as
interoccasion variability. Statistical model chosen.

Goodness of fit graphs for the assessment of random effects structures
employed.

\hypertarget{covariate-model}{%
\subsection{Covariate Model}\label{covariate-model}}

Method for identifying candidate covariates, description of covariate
entered/excluded at each step and final covariate model selected.

\hypertarget{final-model}{%
\subsection{Final Model}\label{final-model}}

It is advisable to include a diagram or table depicting the various
modeling steps, with the corresponding objective function values.

It is essential to include a table with all final model parameter
estimates. The table should include parameter estimates and their
associated uncertainty, with variability reported as CV\% and precisión
reported as the percent relative standard error (RSE\%) or the 95
percent confidence Interval.

Add calculation methods below the table. It is advisable to include a
comparison of parameter estimates from the base to the final model. If a
previous model exists, a comparison of parameter estimates from the
previous and current model may be added. The reliability of the analysis
results can be checked by careful examination of diagnostic plots,
including predicted versus observed concentration, predicted
concentration superimposed on the data, and posterior estimates of
parameter versus covariate values. Checking the parameter estimates,
standard errors, case deletion diagnostics, and sensitivity analysis may
also be appropriate.

\hypertarget{model-evaluation}{%
\section{Model Evaluation}\label{model-evaluation}}

External and/or internal validation including visual predictive checks.
Potential application of bootstrap to obtain confidence intervals

\hypertarget{model-application}{%
\section{Model Application}\label{model-application}}

Model based simulations to evaluate the effect of covariates and other
factors might impact PK (or PD).

\bookmarksetup{startatroot}

\hypertarget{sec-discussion}{%
\chapter{DISCUSSION}\label{sec-discussion}}

\bookmarksetup{startatroot}

\hypertarget{sec-conclusion}{%
\chapter{CONCLUSION}\label{sec-conclusion}}

Summary of major findings. Contextualize the PK model based simulations
with regard to therapeutic window ?

\bookmarksetup{startatroot}

\hypertarget{references}{%
\chapter{REFERENCES}\label{references}}

\hypertarget{refs}{}
\begin{CSLReferences}{0}{0}
\end{CSLReferences}

\cleardoublepage
\phantomsection
\addcontentsline{toc}{part}{Appendices}
\appendix

\hypertarget{si-01}{%
\chapter{SI-01}\label{si-01}}



\end{document}
